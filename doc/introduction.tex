\chapter{Elaborazione progetto}

\section{Introduzione}

Nel seguente documento verranno descritte ed illustrate le fasi di sviluppo del progetto \textbf{Zillow Prize: Zillow’s Home Value Prediction (Zestimate)}.
Nello specifico saranno evidenziate e spiegate le decisioni prese, che hanno portato al completamento dello stesso.
Saranno utilizzati diversi grafici per mostrare alcuni dei dati importanti che hanno portato alle decisioni finali prese.

\section{Creazione del dataframe}
Inizialmente è stato fatto l'import delle diverse librerie e moduli che sono stati utilizzati all'interno del progetto.
Successivamente sono stati caricati i dataset e trasformati in data frame con l'utilizzo della libreria Pandas. Sempre grazie a tale libreria è stato poi fatta 
l'unione tra i due dataset contenenti diverse feature utili, quali '\textbf{logerror}' (la variabile risposta) o la '\textbf{transactiondate}' (data acquisto), così da avere tutte 
le istanze all'interno di un unico data frame.

\section{Pulizia dei dati e feature engeneering}
Stampando il contenuto e il tipo delle varie feature contenute nel data frame appena costruito, è risultato che per la maggior parte erano \textbf{float64} mentre solo alcune erano 
\textbf{object}. Le prime non presentavano problemi nella gestione, mentre per quanto riguarda le seconde è stato necessario effettuare altre operazioni che is vedranno successivamente 
per poterle gestire correttamente.
%parte outliers%
La prima feature che si decise di gestire 

\section{Analisi dataframe post lavorazione dati}

\section{Creazione modelli di regressione e confronti}

\section{Conclusioni}
